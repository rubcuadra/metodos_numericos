\documentclass{report}

\usepackage{graphicx}
\usepackage{listings}
\usepackage{color}
\definecolor{codeColor}{rgb}{0.4, 0.6, 0.8}
\lstset{keywordstyle=\color{codeColor},language=Octave,stepnumber=1,basicstyle=\footnotesize}

\usepackage{enumitem}

\newenvironment{steps}[1]{\begin{enumerate}[label=#1 \arabic*]}{\end{enumerate}}

\makeatletter
\def\step{
   \@ifnextchar[ \@step{\@noitemargtrue\@step[\@itemlabel]}}
\def\@step[#1]{\item[#1]\mbox{}\\\hspace*{\dimexpr-\labelwidth-\labelsep}}
\makeatother

\begin{document}

\title{ Solucion de Ecuaciones No Lineales}
\author{Rub�n Cuadra A01019102}

\maketitle
\begin{abstract}
	Manual de usuario:  'gaussJordan.m', el objetivo de este texto es documentar la implementacion del metodo gaussJordan como metodo para encontrar solucion a ecuaciones asi como calcular inversas.
\end{abstract}

\section{Introduccion}
	El codigo consiste en un archivo llamado de la misma manera que la funcion el cual recibe 3 argumentos y nos regresa 4 respuestas
\lstinputlisting[ firstline=1, lastline=1]{gaussJordan.m}

	\textbf{A}  Es una matriz de nxn que representa los coeficientes del sistema de ecuaciones
		
	\textbf{b} Es un vector de nx1 que es el vector de terminos independientes  

	\textbf{op} Es una bandera True(1) o False(0), cuando es verdaera resuelve el sistema y regresa el vector de soluciones, cuando es falso regresa la inversa y el determinante.

\section{Comprension del algoritmo}
	
\subsection{Gauss-Jordan}
Realiza una serie de operaciones elementales de matrices sobre una matriz extendida, para convertir una matriz de nxn en la identidad, generando asi un vector/matriz resultado  

El algoritmo usado es: 

\begin{steps}{Paso}
\step Inicializar retornos
\step Validamos que sea una matriz nxn 
\step Calculamos determinante usando $det = [A*adj(A)][i,i]$
\step Si no es invertible regresar 
\step IF para ver si regresaremos Inversa o la Solucion del sistema de ecuaciones
\step Ambos casos realizan eliminacion Gausseana, uno se manda a llamar con el vector A y \textbf{b} y otra con A y la identidad
\step Generamos matriz expandida
\step Usamos cambios de renglon para quitar todos los 0s de la diagonal principal
\step Se obtiene el elemento i,i y se realiza multiplicacion de escalar y suma de renglon para convertir en 0 los numeros debajo de el
\step Cuando se llega a i=m debemos realizar el paso anterior pero ahora con los elementos sobre la diagonal
\step  En este punto tenemos una diagonal y todo lo demas en 0, realizamos multiplicacion de escalar por renglon para convertir el numero de la diagonal en 0
\step  A esta altura la parte expandida se vio afectada por las operaciones y tendremos como resultado la matriz inversa o en su defecto, un vector con resultados del sistema
\step  Regresar valores de exito

\end{steps}
Al final se regresan 4 valores,   \textbf{x} ,  \textbf{matrizInversa},  \textbf{d},  \textbf{solucion}
\section{Ejemplo}
Mandamos a llamar la funcion desde un archivo main.m.
Usando los argumentos: 
\lstinputlisting[language=Octave,firstline=1,lastline=6]{main.m}
Llamamos la funcion:
\lstinputlisting[language=Octave,firstline=7,lastline=7]{main.m}
y nos devolveria
\lstinputlisting[language=Octave,firstline=9,lastline=15]{main.m}

\section{Conclusion}
Es un modulo portable que posee varias excepciones, esta facil de implementar y todo bien bien comentado
\footnotetext{Los acentos no se pudieron agregar por cuestion de la codificacion}

\end{document}
